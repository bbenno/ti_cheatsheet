\documentclass[a4paper,parskip=half,footsepline=on,headings=normal,titlepage=false]{scrartcl}
\usepackage[utf8]{inputenc}
\usepackage[T1]{fontenc}
\usepackage[ngerman]{babel}
\usepackage{hyperref}
\usepackage{csquotes}
\usepackage[nolist]{acronym}
\usepackage{microtype}
\usepackage{mathtools}
\usepackage{amssymb}
\usepackage{enumitem}
\usepackage{xcolor}

\definecolor{c1}{RGB}{39, 208, 242}
\definecolor{c2}{RGB}{51, 60, 242}
\definecolor{c3}{RGB}{52, 242, 27}
\definecolor{c4}{RGB}{242, 114, 2}
\definecolor{c5}{RGB}{245, 197, 0}

\usepackage{tikz}
\usetikzlibrary{shapes.geometric}

\title{Cheatsheet}
\subtitle{\acl{TI} --- Prof.~Mauerer --- IN,IM,IT --- \acs{WiSe}21/22}
\author{Benno Bielmeier}
%\date{\ac{WiSe}~2020}

% Repräsentation von Nicht-Terminal
\newcommand{\nt}[1]{\langle \uppercase{#1} \rangle}

\setlist{noitemsep}     % no separation space between enum items

\setuptoc{toc}{leveldown}

\setcounter{tocdepth}{1}

%\RedeclareSectionCommand[
  %runin=false,
  %afterindent=false,
  %beforeskip=\baselineskip,
  %afterskip=.5\baselineskip]{section}
\RedeclareSectionCommand[
  %runin=false,
  %afterindent=false,
  %beforeskip=.75\baselineskip,
  afterskip=0.1\baselineskip]{subsection}

\renewcommand{\epsilon}{\varepsilon}

% each section on its own page
\addtokomafont{section}{\clearpage}

\begin{document}

%\numberwithin{equation}{section}

\begin{acronym}
    \acro{TI}{Theoretische Informatik}
    \acro{SoSe}{Sommersemester}
    \acro{WiSe}{Wintersemester}
    \acro{PZ}{Pumping Zahl}
    \acro{PL}{Pumping Lemma}
    \acro{DEA}{deterministischer endlicher Automat}
    \acro{NEA}{nicht-deterministischer endlicher Automat}
\end{acronym}
\acused{SoSe}
\acused{WiSe}

\maketitle
\tableofcontents

\section{\texorpdfstring{$\epsilon$}{Epsilon}-Elimination einer Grammatik}
Vorgehen:
\begin{enumerate}[label=\alph*)]
    \item Suche alle Regel der Form {\color{c1}$\nt{\dots} \to \epsilon$}
    \item {\color{c2}Eliminiere $\epsilon$}, durch Einsetzen ebendieser Regeln für alle Vorkommnisse von \textcolor{c1}{$\nt{\dots}$}
    \item {\color{c3}Isoliere $\epsilon$} mittels neuem Nicht-Terminal \textcolor{c3}{$\nt{S'}$}
\end{enumerate}

\subsection*{Beispiel}

\begin{subequations}
\begin{align}
    \label{eq:epsilon:1_search}
    \begin{split}
    \color{c1}\nt{S} & \to a \nt{S} b \mid \nt{B} \mid \color{c1}\epsilon \\
    \color{c1}\nt{A} & \to a \mid \nt{A}e \mid \color{c1}\epsilon \\
    \nt{B} & \to b\nt{A}c
    \end{split}
    \\ \nonumber \\
    \label{eq:epsilon:2_elemination}
    \begin{split}
    \nt{S} & \to a \nt{S} b \mid \epsilon \mid \nt{B} \mid a{\color{c2}\epsilon}b \\
    \nt{A} & \to a \mid \nt{A}e \mid {\color{c2}\epsilon}e \\
    \nt{B} & \to b\nt{A}c \mid b{\color{c2}\epsilon}c
    \end{split}
    \\ \nonumber \\
    \label{eq:epsilon:3_isolation}
    \begin{split}
    \color{c3}\nt{S} & \color{c3}\to \epsilon \mid \nt{S'} \\
    \nt{S'} & \to a \nt{S} b \mid ab \\
    \nt{A} & \to a \mid e \\
    \nt{B} & \to b\nt{A}c \mid bc
    \end{split}
\end{align}
\end{subequations}

\section{Beweis mittels \acl{PL}}
Mit dem \ac{PL} kann lediglich bewiesen werden, dass eine Sprache $L$ nicht regulär ist.
Die Beweisführung für $L \in \mathcal{L}$ erfolgt über andere Verfahren\footnote{Satz von \emph{Myhill und Nerode}, \emph{\acs{DEA}}- bzw.\ \emph{\acs{NEA}}-Konstruktion, Abbildung über \emph{reguläre Grammatik}, Finden eines \emph{regulären Ausdrucks}, \emph{Abschlusseigenschaften} von regulären Sprachen}.

\subsection{Annahme / Setup}
Annahme: $L \in \mathcal{L}_3$ \\
$\implies \exists \text{ \acl{PZ} } n \in \mathbb{N}$ sodass $\forall w \in L$ mit $|w| \ge n$ zerlegbar in $xyz$, sodass
\begin{enumerate}
    \item $|xy| \le n$
    \item $|y| \ge 1$
    \item $\forall i \in \mathbb{N}_0 : xy^iz \in L$
\end{enumerate}

\subsection{Wort-Wahl}
Wähle ein Wort $w_0$ und zeige 
\begin{itemize}
    \item $w_0 \in L$
    \item $|w_0| \ge n$
\end{itemize}

\subsection{Regelanwendung}
Anwenden der drei Regeln auf $w_0$ und finden eines Widerspruchs.
Bei Fallunterscheidung müssen \emph{alle} Optionen auf einen Widerspruch geführt werden.

\subsection{Widerspruch / Abschluss}
Widerspruch (z.B.\ auf-/abgepumtes Wort $w' \notin L$)
\begin{itemize}[leftmargin=2.0cm,label=$\implies$]
    \item nicht alle Aussagen des \ac{PL} zutreffend
    \item Annahme ($L \in \mathcal{L}_3$) falsch
    \item $L \notin \mathcal{L}_3$
\end{itemize}

\section{Table-Filling-Algorithmus}
$\rightarrow$ Minimal-Automat ($|R_L| = |R_M|$) für Reguläre Sprachen

\begin{enumerate}
    \item Erstelle Tabelle
    \item Markiere alle Paare $(q, q')$ mit $q\in F \wedge q\notin F$
    \item Teste $\forall$ unmarkierte Paare $(q, q')$ und $\forall \sigma \in \Sigma$
    \[
     (\delta(q,\sigma_i), \delta(q',\sigma_i))
      \begin{cases} 
       \text{bereits markiert } & \implies \text{markiere } (q, q') \\
       \text{nicht markiert } & \implies \text{do nothing}
      \end{cases}
    \]
    \item Wiederhole Schritt 3 bis Tabelle invariant
    \item Verschmelze alle nicht markierten Paare
\end{enumerate}

\end{document}
